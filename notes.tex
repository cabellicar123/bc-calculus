\documentclass[12pt]{article}

\usepackage[utf8]{inputenc}
\usepackage[T1]{fontenc}
\usepackage{lmodern}
\usepackage[fleqn]{mathtools}
\usepackage{amssymb}
\usepackage[margin=1.0in]{geometry}
\usepackage{parskip}
\usepackage[compact]{titlesec}

\setlength{\parindent}{0pt}
\setlength{\mathindent}{0pt}

\titleformat*{\subsection}{\bfseries}

\let\oldimplies\implies%
\renewcommand*{\implies}{\Rightarrow}
\newcommand*{\R}{\mathbb{R}}
\newcommand*{\rmaskip}{\setlength{\abovedisplayskip}{0pt}}
\newcommand*{\rmbskip}{\setlength{\belowdisplayskip}{0pt}}
\newcommand*{\rmskip}{\rmaskip \rmbskip}
\newcommand*{\isin}{\sin^{-1}}
\newcommand*{\icos}{\cos^{-1}}
\newcommand*{\itan}{\tan^{-1}}
\newcommand*{\dd}[3][]{\tfrac{\mathrm{d}^{#1}#2}{\mathrm{d}#3^{#1}}}
\newcommand*{\D}[1]{\mathop{}\!\mathrm{d} #1}
\newcommand*{\isum}[1][k]{\sum_{#1=0}^\infty}
\makeatletter
\newcommand{\mquote}[1]{{\mathpalette\mqu@te{#1}}}
\newcommand{\mqu@te}[2]{%
  \sbox0{\(\m@th#1\text{``}\)}%
  \sbox1{\(\m@th#1\text{''}\)}%
  \sbox2{\(\m@th#1#2\)}%
  \ifdim\ht2>\ht0
    \raisebox{\dimexpr\ht2-\height}{\box0}%
    #2%
    \raisebox{\dimexpr\ht2-\height}{\box1}%
  \else
    \box0 #2\box1
  \fi
}
\makeatother

\begin{document}
\section*{Double Angle Formulas}
{\rmskip\begin{align*}
  \sin(2x) &= 2\sin(x)\cos(x)\\
  \cos(2x) &= \cos^2(x) - \sin^2(x)\\
    &= 2\cos^2(x) - 1\\
    &= 1-2\sin^2(x)\\
  \tan(2x) &= \frac{2\tan(x)}{1 - \tan^2(x)}\\
  \sin^2(x) &= \frac{1}{2}(1-\cos(2x))\\
  \cos^2(x) &= \frac{1}{2}(1+\cos(2x))
\end{align*}}%
\section*{Limits}
\[
  \lim_{x \to n} f(x) = L
\]
Means for any number \(\epsilon > 0\) there exists a number \(\delta > 0\) such
that if \(0 < |x - n| < \delta\) then \(|f(x) - L| < \epsilon\).
\[
  \lim_{x \to a}f(g(x)) = f\left(\lim_{x \to a}g(x)\right) \text{ if } f(g(x))
  \text{ is continuous at } x
\]
Standard difference quotient:
\[
  f'(x) = \lim_{h \to 0}\frac{f(x+h) - f(x)}{h}
\]
Symmetric difference quotient:
\[
  f'(x) = \lim_{h \to 0}\frac{f(x+h) - f(x-h)}{2h}
\]
\section*{Continuity and Differentiability}
\(f(x)\) is continuous at \(x\) if:
\[
  f(x) = \lim_{k \to x^+}f(k) = \lim_{k \to x^-}f(k)
\]
\(f(x)\) is differentiable at \(x\) if:
\[
  f'_+(x) = f'_-(x)
\]
\section*{Differentiation Rules}
{\rmskip\begin{align*}
  \dd{}{x}c f(x) &= c f'(x)\\
  \dd{}{x}f(g(x)) &= f'(g(x))g'(x)\\
  \dd{}{x}(f(x) \pm g(x)) &= f'(x) \pm g'(x)\\
  \dd{}{x}{f(x)}^n &= n{f(x)}^{n-1} \cdot f'(x)\\
  \dd{}{x}\sqrt{f(x)} &= \frac{f'(x)}{2\sqrt{f(x)}}\\
  \dd{}{x}(f(x)g(x)) &= f'(x)g(x) + f(x)g'(x)\\
  \dd{}{x}\left(\frac{f(x)}{g(x)}\right) &= \frac{f'(x)g(x) - f(x)g'(x)}
  {{g(x)}^2}\\
  \dd{}{x}|f(x)| &= \frac{|f(x)|}{f(x)} \cdot f'(x)\\
  \dd{}{x}b^{f(x)} &= b^{f(x)} \ln b \cdot f'(x)\\
  \dd{}{x}\log_b(f(x)) &= \frac{f'(x)}{f(x) \ln b}
\end{align*}}%
\section*{Trigonometric Differentiation Rules}
{\rmskip\begin{align*}
  \dd{}{x}\sin(f(x)) &= \cos(f(x)) \cdot f'(x)\\
  \dd{}{x}\cos(f(x)) &= -\sin(f(x)) \cdot f'(x)\\
  \dd{}{x}\tan(f(x)) &= \sec^2(f(x)) \cdot f'(x)\\
  \dd{}{x}\sec(f(x)) &= \sec(f(x))\tan(f(x)) \cdot f'(x)\\
  \dd{}{x}\csc(f(x)) &= -\csc(f(x))\cot(f(x)) \cdot f'(x)\\
  \dd{}{x}\cot(f(x)) &= -\csc^2(f(x)) \cdot f'(x)
\end{align*}}%
\section*{Inverse Trigonometric Differentiation Rules}
{\rmskip\begin{align*}
  \dd{}{x}\isin(f(x)) &= \frac{f'(x)}{\sqrt{1 - {f(x)}^2}}\\
  \dd{}{x}\icos(f(x)) &= -\frac{f'(x)}{\sqrt{1 - {f(x)}^2}}\\
  \dd{}{x}\itan(f(x)) &= \frac{f'(x)}{1 + {f(x)}^2}
\end{align*}}%
\section*{Mean Value Theorem}
If \(f(x)\) is continuous on \([a,b]\), then there exists one value \(c\) on
\([a,b]\) such that:
\[
  f'(c) = \frac{f(b) - f(a)}{b - a}
\]
\section*{Reimann Sums}
\[
  \sum_{k=1}^n f(x_k) \Delta x\\
\]
Assuming regular partition:
{\rmbskip\begin{align*}
  \text{LRAM} &= \sum_{k=1}^n f\left(a + \frac{(k-1)(b - a)}{n}\right)
    \frac{b-a}{n}\\
  \text{MRAM} &= \sum_{k=1}^n f\left(a + \frac{(k-0.5)(b - a)}{n}\right)
    \frac{b-a}{n}\\
  \text{RRAM} &= \sum_{k=1}^n f\left(a + \frac{k(b - a)}{n}\right)
    \frac{b-a}{n}
\end{align*}}%
\section*{Definite Integration}
{\rmaskip\begin{align*}
  \int_a^b f(x) \D{x} &= \lim_{n \to \infty} \sum_{k=1}^n f(x_k)\Delta x\\
  \int_a^a f(x) \D{x} &= 0\\
  \int_a^b f(x) \D{x} &= -\int_b^a f(x) \D{x}\\
  \int_a^b c f(x) \D{x} &= c \int_a^b f(x) \D{x}\\
  \int_a^b (f(x) \pm g(x)) \D{x} &= \int_a^b f(x) \D{x} \pm \int_a^b g(x)
    \D{x}\\
  \int_a^p f(x) \D{x} + \int_p^b f(x) \D{x} &= \int_a^b f(x) \D{x}\\
  \int_a^b f(x) \D{x} &> \int_a^b g(x) \D{x} \text{ if }f(x) > g(x)
    \text{ on } [a,b]
\end{align*}}%
The mean value of \(f(x)\) over \([a,b]\) is:
\[
  \frac{1}{b - a} \int_a^b f(x) \D{x}
\]
\section*{Mean Value of a Function Theorem}
If \(f(x)\) is continuous on \([a,b]\), then there exists one value \(c\) on
\([a,b]\) such that:
\[
  f(c) = \frac{1}{b - a} \int_a^b f(x) \D{x}
\]
\section*{Linear Approximation}
The linearization of \(f(x)\) at \(a\):
\[
  L(x) = f(a) + f'(a)(x-a)
\]
\section*{Fundamental Theorem of Calculus}
\subsection*{Part 1}
\[
  \dd{}{x} \int_{a(x)}^{b(x)} f(t) \D{t} = f(b(x)) b'(x) - f(a(x)) a'(x)
\]
\subsection*{Part 2}
{\rmskip\begin{align*}
  F'(x) &= f(x)\\
  \int_a^b f(x) &= F(x) \bigg|_a^b = F(b) - F(a)
\end{align*}}%
\section*{Solving Differential Equations Using A Definite Integral}
{\rmaskip\begin{align*}
  \dd{y}{x} &= f(x)\\
  y &= \int_a^x f(t) \D{t} + C
\end{align*}}%
The easiest solution occurs when \(a = \) the given \(x\) value and \(C = \)
the given \(y\) value.
\section*{Solving Differential Equations Using Antidifferentiation}
{\rmaskip\begin{align*}
  \dd{y}{x} &= f(x)\\
  y &= \int f(x) + C \text{ (general form)}
\end{align*}}%
Specify largest domain of \(x\) that includes specified \(x\) and does not
include any discontinuities.
\section*{Euler's Method}
Start at \((x_0,y_0)\)
{\rmbskip\begin{align*}
  \Delta y &= \Delta x \cdot \dd{y}{x} \biggr|_{(x_{n-1},y_{n-1})} \\
  (x_n.y_n) &= (x_{n-1} + \Delta x, y_{n-1} + \Delta y)
\end{align*}}%
\section*{Antidifferentiation Rules}
{\rmskip\begin{align*}
  \int cf(x) \D{x} &= c \int f(x) \D{x}\\
  \int f(x) \pm g(x) \D{x} &= \int f(x) \D{x} \pm \int g(x) \D{x}\\
  \int x^n \D{x} &= \frac{x^{n+1}}{n+1} + C\\
  \int \frac{1}{x} \D{x} &= \ln|x| + C\\
  \int b^x \D{x} &= \frac{b^x}{\ln b} + C\\
  \int \ln x \D{x} &= x\ln x - x + C
\end{align*}}%
\section*{Trigonometric Antidifferentiation Rules}
{\rmskip\begin{align*}
  \int \sin(x) \D{x} &= -\cos(x) + C\\
  \int \cos(x) \D{x} &= \sin(x) + C\\
  \int \tan(x) \D{x} &= -\ln|\cos(x)| + C\\
  \int \sec(x) \D{x} &= \ln|\sec(x)+\tan(x)| + C\\
  \int \sin^2(x) \D{x} &= \frac{1}{2} (x - \sin(x)\cos(x)) + C\\
  \int \cos^2(x) \D{x} &= \frac{1}{2} (x + \sin(x)\cos(x)) + C\\
  \int \tan^2(x) \D{x} &= \tan(x) - x + C\\
  \int \sec^2(x) \D{x} &= \tan(x) + C\\
  \int \sec(x) \tan(x) \D{x} &= \sec(x) + C\\
  \int \frac{1}{1+x^2} \D{x} &= \itan(x) + C\\
  \int \frac{1}{\sqrt{1-x^2}} \D{x} &= \isin(x) + C
\end{align*}}%
\section*{Inverse Trigonometric Antidifferentiation Rules}
{\rmskip\begin{align*}
  \int \isin(x) &= x\isin(x) + \sqrt{1-x^2}\\
  \int \itan(x) &= x\itan(x) - \frac{1}{2} \ln(1 + x^2) + C
\end{align*}}%
\section*{U-Substitution}
To differentiate \(\int f(g(x)) \D{x}\) where \(f(x)\) and \(g(x)\) are both
antidifferentiable, set \(u = g(x)\). Then solve for \(du\) and replace \(dx\)
so that the equation \(A\int f(u) \D{u}\) results (where \(A\) is the necessary
``fudge factor''). Antidifferentiate and replace \(u\) with \(g(x)\).
\section*{Integration by Parts}
\[
  \int u \D{v} = uv - \int v \D{u} + C
\]
or\\
{\renewcommand{\arraystretch}{1.5}\begin{tabular}{c c}
  \(u\) & \(\D{v}\)\\
  \(\dd{u}{x}\) & \(\int \D{v} \D{x}\)\\
  \(\dd[2]{u}{x}\) & \(\int\left(\int\D{v}\D{x}\right)\D{x}\)\\
  \(\dd[3]{u}{x}\) & \(\int\left(\int\left(\int\D{v}\D{x}\right)\D{x}
    \right)\D{x}\)\\
  \(\cdots\) & \(\cdots\)\\
  0 & \(\cdots\)
\end{tabular}}
\[
  \int u \D{v} = \left(u \cdot \int \D{v} \D{x}\right) - \left(
    \dd{u}{x} \cdot \int \left( \int \D{v} \D{x} \right) \D{x}
    \right) + \left(\dd[2]{u}{x} \cdot \int \left( \int \left( \int \D{v}
    \D{x} \right) \D{x} \right) \D{x} \right) - \dots
\]
\section*{Partial Fraction Decomposition}
Convert fractions of form \(\frac{\text{linear or constant}}{\text{factorable
quadratic}}\) into its partial fraction.
\begin{align*}
  \frac{h(x)}{f(x)g(x)} &= \frac{A}{f(x)} + \frac{B}{g(x)} = \frac{Ag(x)}
  {f(x)g(x)} + \frac{Bf(x)}{f(x)g(x)} = \frac{Ag(x) + Bf(x)}{f(x)g(x)}\\
  h(x) &= Ag(x) + Bf(x)
\end{align*}
Choose a value of \(x\) such that \(f(x) = 0\) and then \(g(x) = 0\) to solve
for \(A\) and \(B\) respectively.
\section*{Exponential Growth\slash Decay}
{\rmskip\begin{align*}
  \dd{Q}{t} &= kQ\\
  Q &= Q_0 e^{kt} = Q_0 b^{\frac{t}{p}}\\
  k &= -\frac{\ln(b)}{p}
\end{align*}}%
\subsection*{Half-life}
\[
  k = -\frac{\ln(2)}{p}
\]
\section*{Logistic Differential Equation}
\[
  \dd{P}{t} = kP(m - P) = kPm\left(1-\frac{P}{m}\right)
\]
where \(P\) is the population and \(M\) is the carrying capacity
\[
  P(t) = \frac{m}{1 + Ae^{-mkt}}
\]
The maximum carrying capacity is reached when \(P(t) = m - 0.5\), if \(P(0) <
m\), or right after \(P(t) = m + 0.5\), if \(P(0) > m\).
\section*{Calculating Area of Regions Bounded by Two Functions}
\[
  A = \int_a^b f(x) - g(x) \D{x}
\]
where \(f(x)\) is the upper function and \(g(x)\) is the lower function.

If the functions cross, multiple integrals must be taken with the appropriate
upper and lower functions.
\section*{Typical Cross-Sectional Areas}
{\renewcommand{\arraystretch}{1.5}\begin{tabular}{l@{ }l}
  Square: & \(A = b^2 = \frac{1}{2}d^2\)\\
  Rectangle: & \(A = bh\)\\
  Equalateral triangle: & \(A = \frac{\sqrt{3}}{4}b^2\)\\
  Right isoceles triangle: & \(A = \frac{1}{2}l^2 = \frac{1}{4}h^2\)\\
  Hexagon: & \(A = \frac{3\sqrt{3}}{2} s^2\)\\
  Circle: & \(A = \pi r^2 = \frac{1}{4}\pi d^2\)\\
  Semicircle: & \(A = \frac{1}{2}\pi r^2 = \frac{1}{8}\pi d^2\)\\
  Parabolic region: & \(A = \frac{2}{3}bh\)
\end{tabular}}
\section*{Calculating Volume Using the Disk\slash Washer Method for Solids of
Rotation}
\[
  V = \pi\int_a^b r_{\text{outer}}^2 - r_{\text{inner}}^2 \D{t}
\]
where \(\D{t}\) is the differential thickness of the disk\slash washer.
\section*{Calculating Volume Using the Cylindrical Shell Method for Solids of
Rotation}
\[
  V = 2\pi \int_a^b hr \D{t}
\]
where \(\D{t}\) is the differential thickness of the shell.
\section*{Calculating Arc Lengh}
\[
  l = \int_a^b \sqrt{1+{\left(\dd{y}{x}\right)}^2} \D{x}
  = \int_c^d \sqrt{1+{\left(\dd{x}{y}\right)}^2} \D{y}
\]
\section*{L'H\^{o}pitals Rule}
In order to evaluate a limit where
\[
  \lim_{x \to a} \frac{f(x)}{g(x)} \implies \mquote{\frac{0}{0}} \text{ or }
  \mquote{\frac{\infty}{\infty}}
\]
evaluate
\[
  \lim_{x \to a} \frac{f'(x)}{g'(x)}
\]
In order to evaluate a limit where
\[
  \lim_{x \to a} {f(x)}^{g(x)} \implies \mquote{0^0} \text{ or }
  \mquote{1^\infty} \text{ or } \mquote{\infty^0}
\]
set \(L\) equal to the original limit and immediately take the log of both
sides
\begin{align*}
  L &= \lim_{x \to a} {f(x)}^{g(x)}\\
  \ln(L) &= \ln\left(\lim_{x \to a} {f(x)}^{g(x)}\right)\\
  \ln(L) &= \lim_{x \to a} g(x) \ln f(x)
\end{align*}
Evaluate the resulting limit and then raise \(e\) to that result
\section*{Relative Rates of Growth}
To compare rates of \(f(x)\) and \(g(x)\) evaluate
\[
  \lim_{x \to \infty} \frac{f(x)}{g(x)} =
  \begin{dcases*}
    \infty & then \(f(x)\) grows faster than \(g\)\\
    \left<0, \infty\right> & then \(f(x)\) and \(g(x)\) grow at the same rate\\
    0 & then \(f(x)\) grows slower than \(g(x)\)
  \end{dcases*}
\]
\section*{Improper Integrals}
\subsection*{Type 1}
{\rmskip\begin{align*}
  \int_a^\infty f(x) \D{x} &= \lim_{n \to \infty} \int_a^n f(x) \D{x}\\
  \int_{-\infty}^b f(x) \D{x} &= \lim_{n \to \infty} \int_{-n}^b f(x) \D{x}\\
  \int_{-\infty}^\infty f(x) \D{x} &= \lim_{n \to \infty} \int_{-n}^n f(x) \D{x}
\end{align*}}%
\subsection*{Type 2}
\[
  \int_a^b f(x)
\]
such that there is at least one vertical-asymptotic discontinuity on \([a,b]\).
\begin{alignat*}{3}
  &\int_a^b f(x) \D{x} \text{ (where discontinuity at \(a\)) }
    &&= \lim_{n \to a^+} \int_n^b f(x) \D{x}\\
  &\int_a^b f(x) \D{x} \text{ (where discontinuity at \(b\)) }
    &&= \lim_{n \to b^-} \int_a^n f(x) \D{x}\\
  &\int_a^b f(x) \D{x} \text{ (where discontinuity \(d\) on
    \(\left<a,b\right>\)) } &&= \lim_{n \to d^-} \int_a^n f(x) \D{x} +
    \lim_{n \to d^+} \int_n^b f(x) \D{x}
\end{alignat*}
\subsection*{P-Integrals}
An integral that fits the pattern:
\[
  \int_1^\infty \frac{1}{x^p} \D{x} \text{ where \(p\) is positive}
\]
{\rmaskip\begin{align*}
  p > 1 &\implies \text{the integral converges to \(\tfrac{1}{p-1}\).}\\
  0 < p \le 1 &\implies \text{the integral diverges.}
\end{align*}}%
\section*{Comparison Tests}
For type 1 improper integrals (\(a = \infty\) and\slash or \(b = \infty\))
where the integrand is not easily antidifferentiable:
\[
  \int_a^b f(x) \D{x}
\]
Establish bound functions which can be antidifferentiated:
\[
  l(x) \le f(x) \le u(x)
\]
{\rmskip\begin{alignat*}{3}
  \text{Convergence if } &\int_a^b f(x) \D{x} \le \int_a^b u(x) \D{x}
    &&\text{ and } \int_a^b u(x) \D{x} \text{ converges}\\
  \text{Divergence if } &\int_a^b f(x) \D{x} \ge \int_a^b l(x) \D{x}
    &&\text{ and } \int_a^b l(x) \D{x} \text{ diverges}
\end{alignat*}}%
\section*{Sequences}
\subsection*{Explicit Definition}
\[
  x_k = f(k)
\]
where \(k \in \{1,2,3,\dots\}\).
\subsection*{Recursive Definition}
\[
  x_k = m(x_{k-1})
\]
where \(x_1\) is defined and \(k \in \{2,3,4,\dots\}\).
\subsection*{Arithmetic}
{\rmskip\begin{align*}
  x_k &= d(k-1) + x_1\\
  x_k &= t_{k-1} + d
\end{align*}}%
\subsection*{Geometric}
{\rmskip\begin{align*}
  x_k &= x_1r^{k-1}\\
  x_k &= x_{k-1} r
\end{align*}}%
\subsection*{Convergence\slash Divergence}
A sequence converges if the limit \(\displaystyle\lim_{n \to \infty} x_n\) is
a finite number else it diverges.
\section*{Infinite Series}
\[
  \isum x_k
\]
\subsection*{Convergence}
Consider the sequence of ``partial sums'' (\(s\)) where:
\[
  s_n = \sum_{k=0}^n x_k
\]
If \(\{s_n\}\) converges to \(S\) then the infinite series converges to \(S\).
\subsection*{Power Series}
\[
  \isum c_k{(x-a)}^k
\]
\subsection*{Taylor/Maclaurin Series}
Given an infinitely differentiable function \(f(x)\), the Taylor series
centered around \(a\) is equal to the power series with a specific \(c_k\):
\[
  f(x) = \isum \frac{f^{(k)}(a)}{k!}{(x-a)}^k
\]

A Maclaurin series is a Taylor series centered around 0 (\(a = 0\)).

The \(n\)-order polynomial function \(P_n(x)\) that best fits \(f(x)\) is
equal to the Taylor series evaluated with upper limit \(n\):
\[
  P_n(x) = \sum_{k=0}^{n} \frac{f^{(k)}(a)}{k!}{(x-a)}^k
\]
\subsection*{Common Maclaurin Series}
{\rmskip\begin{alignat*}{3}
  \frac{a}{1-x} &= \isum ax^k &&\quad x \in \left<-1,1\right>\\
  e^x &= \isum \frac{x^k}{k!} &&\quad x \in \R\\
  \sin(x) &= \isum {(-1)}^k \frac{x^{2k+1}}{(2k+1)!} &&\quad x \in \R\\
  \cos(x) &= \isum {(-1)}^k \frac{x^{2k}}{(2k)!} &&\quad x \in \R\\
  \itan(x) &= \isum {(-1)}^k \frac{x^{2k+1}}{2k+1} &&\quad x \in [-1,1]\\
  \ln(x+1) &= \sum_{k=1}^\infty {(-1)}^{k+1} \frac{x^k}{k} &&\quad x \in
    \left<-1,1\right]
\end{alignat*}}%
\subsection*{Calculating Error for a Taylor Polynomial}
\[
  P_\infty(x) = P_n(x) + R_n(x)
\]
where \(P_\infty\) is the entire infinite Taylor series and \(R_n\) is the
remainder.
\[
  |P_n(x) - f(x)| = \text{magnitude of error}
\]
\end{document}
